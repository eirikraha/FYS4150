\documentclass[12pt,a4paper]{article}
\usepackage[utf8]{inputenc}
\usepackage[english]{babel} 
\usepackage{fullpage} 
\usepackage{amsmath} 
\usepackage{amsfonts} 
\usepackage{amssymb} 
\usepackage{verbatim} 
%\usepackage{listings} 
\usepackage{color} 
\usepackage{setspace} 
\usepackage{epstopdf} 
\usepackage{braket}
\usepackage{graphicx} 
\usepackage{float}
\usepackage{cite} 
\usepackage{url}
\usepackage[makeroom]{cancel}
\usepackage{caption}
\usepackage{subcaption}
\pagestyle{empty}
\usepackage[procnames]{listings}
\bibliographystyle{plain}

\makeatletter
\newcommand*{\rom}[1]{\expandafter\@slowromancap\romannumeral #1@}
\makeatother

\newcommand{\s}{^{*}}
\newcommand{\V}[1]{\textbf{#1}}

% Some commands etc. added by Simen
%\usepackage{physics}
\usepackage{units}
\newcommand\smat[1]{\big(\begin{smallmatrix}#1\end{smallmatrix}\big)}
\newcommand\ppmat[1]{\begin{pmatrix}#1\end{pmatrix}}
\newcommand\numberthis{\addtocounter{equation}{1}\tag{\theequation}}
% Usage: \numberthis \label{name}
% Referencing: \eqref{name}


\author{Eirik Ramsli Hauge}
\title{Project 1 in FYS4150}

\begin{document}
	\maketitle
	
\begin{abstract}
This is a place to write something really deep. Like: The future is a really big place. Wow...
\end{abstract}
\subsection*{Introduction}
\subsection*{Theory}
\begin{align*}
\V{A}\times \V{v} &= \V{p} \\
\begin{pmatrix}
b_1 & c_1 & 0 & 0 \\
a_2 & b_2 & c_2 & 0 \\
0 & a_3 & b_3 & c_3 \\
0 & 0 & a_4 & b_4 \\
\end{pmatrix}
\begin{pmatrix}
v_1 \\ v_2 \\ v_3 \\ v_4
\end{pmatrix} &= 
\begin{pmatrix}
p_1 \\ p_2 \\ p_3 \\ p_4
\end{pmatrix}
\end{align*}
\begin{align*}
\intertext{This gives us the following equations:}
\text{\rom{1}}:& \, v_1b_1 + c_1v_2 &= p_1 \\
\text{\rom{2}}:& \, a_2v_1 + b_2v_2 + c_2v_3 &= p_2 \\
\text{\rom{3}}:& \, a_3v_2 + b_3v_3 + c_3v_4 &= p_3 \\
\text{\rom{4}}:& \, a_4v_3 + b_4v_4 &= p4 \\
\end{align*}
We want only zeroes on the left side of the diagonal. Therefore, we do as follows:
\begin{align*}
p_2\s &= p_2 - p_1\cdot \frac{a_2}{b_1} \\
&= a_2v_1 + b_2v_2 + c_2v_3 - v_1a_2 - v_2\frac{c_1a_2}{b_1} \\
&= v_2(b_2 - \frac{c_1a_2}{b_1}) +c_2v_3 \\
&= v_2b_2\s + c_2v_3 \\
\intertext{Now we have the following matrix:}
\hat{\V{A}} &= \begin{pmatrix}
b_1 & c_1 & 0 & 0 \\
0 & b_2\s & c_2 & 0 \\
0 & a_2 & b_3 & c_3 \\
0 & 0 & a_3 & b_4 \\
\end{pmatrix}
\intertext{As we can see, the a disappers from the second row. Just as we wanted. We do the same for the other rows}
p_3\s &= p_3 - p_2\* \cdot \frac{a_3}{b_2\s} \\
&= v_3(b_3 - \frac{c_2a_3}{b_2\s}) + c_3v_4 \\
&= v_3b_3\s + c_3v_3
\end{align*}
General:
\begin{equation}
p_n\s = p_n - p_{n-1}\cdot\frac{a_{n}}{b_{n-1}\s}
\end{equation}
\begin{equation}
b_n\s = b_n - \frac{c_{n-1}a_{n}}{b_{n-1}\s}
\end{equation}
\begin{equation}
v_n = \frac{p_n\s - c_n v_{n+1}}{b_n\s}
\end{equation}
\subsection*{Programs}
\subsection*{Results}
\subsection*{Discussion}
\subsection*{Conclusion}
\end{document}
